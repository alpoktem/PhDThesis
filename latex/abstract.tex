\selectlanguage{english}
\section*{\Large \sffamily Abstract}

In this dissertation, I study the inclusion of prosody into two applications that involve speech understanding:~automatic speech transcription and spoken language translation. In the former case, I propose a method that uses an attention mechanism over parallel sequences of prosodic and morphosyntactic features. Results indicate an $F_1$ score of 70.3\% in terms of overall punctuation generation accuracy. In the latter problem I deal with enhancing spoken language translation with prosody. A neural machine translation system trained with movie-domain data is adapted with pause features using a prosodically annotated bilingual dataset. Results show that prosodic punctuation generation as a preliminary step to translation increases translation accuracy by 1\% in terms of BLEU scores. Encoding pauses as an extra encoding feature gives an additional 1\% increase to this number. The system is further extended to jointly predict pause features in order to be used as an input to a text-to-speech system. 

\textbf{Keywords:} prosody, automatic speech transcription, punctuation restoration, spoken language machine translation, bilingual spoken corpus

\selectlanguage{catalan}
\vspace*{\fill}
\section*{\Large \sffamily  Resum}

En aquesta tesi estudio la inclusió de la prosòdia en dues aplicacions que involucren la comprensió de la parla:~la transcripció automàtica de la parla i la traducció de la llengua oral. En el primer cas, proposo un mètode que utilitza un mecanisme d’atenció sobre seqüències paral·leles de característiques prosòdiques i morfosintàctiques. Els resultats indiquen una precisió de $F_1$=70.3\% en la generació de la puntuació. En el segon cas m'ocupo de la millora de la traducció de la llengua oral utilitzant la prosòdia. Un sistema neural de traducció automàtica format amb un corpus de text en el domini del cinema s’adapta amb característiques de pauses afegides utilitzant un conjunt de dades bilingües prosòdicament anotada. Els resultats mostren que la generació de puntuació prosòdica com a pas previ a la traducció augmenta la precisió de la traducció en un 1\% en termes de BLEU. La codificació de les pauses com a característica addicional encara incrementa la precisió en un altre 1\%. A més a més, amplio el sistema de traducció per a predir conjuntament les característiques de pausa i poder-les utilitzar com a entrada en un sistema de síntesi de veu. 

\textbf{Paraulas clau:} prosòdia, transcripció automàtica de la parla, restauració de la puntuació, traducció automàtica de llenguatge oral, corpus bilingües

\vspace*{\fill}

\selectlanguage{English}



