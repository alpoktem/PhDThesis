\selectlanguage{english}
\section*{\Large \sffamily Abstract}

In the current era where humans build more and more interaction with machines, prosody still is a less-focused phenomena in systems with natural language understanding. Automatic speech recognition systems give focus on ``what is said?'' and disregard ``how it is said?''. In the same way, spoken language translation systems that are built on this kind of “ordinary” recognition, disregard any prosodic information in the input utterance, which might have to be reflected in the translation and synthesis in the target language. Prosody, which is the sum of features such as intonation, stress, and rhythm carried among segments in speech, is an important aspect in human spoken communication. It acts as an essential element from pointing out importance on certain aspects to expressing emotions or to help structure our discourses. In order to arrive to a point where machines can listen to us like humans do, prosody as an essential part of the spoken language needs to be taken into account. 

In this dissertation, we study the inclusion of prosody into two applications that involve speech understanding: automatic speech transcription and spoken language translation. We are specifically motivated by two emerging real-world use cases: automatic subtitling and automatic dubbing. These use cases promise an important share in future machine intelligence applications considering the exponential growth of media. In the former case, we propose methods to improve accuracy of automatically generated punctuation through the use of several morphosyntactic and prosodic features. Punctuation, which proves to be essential both for human and machine understanding, is normally missing in ASR output. Taking into knowledge that punctuation is influenced by both syntax and prosody, we first report on the effect of each prosodic feature besides lexical features to punctuation generation on raw speech transcripts. Through our experiments, the optimal feature set proves to be inter-word pausing, word-level intonation and POS features. Our method that uses an attention mechanism over parallel sequences of prosodic and morphosyntactic features indicate an F1 score of 70.3\% in terms of overall punctuation generation accuracy on manually generated transcripts. Our proposed system is further demonstrated on a state-of-the-art ASR system and shows promising results. 

In the latter problem we deal with enhancing spoken language translation with prosody. We present a novel method for collecting prosodically annotated bilingual datasets and use it to gather an English-Spanish parallel corpus made up of 7000 aligned and transcribed speech segments from original and dubbed version of a TV series. A neural machine translation system is built and trained with a larger movie-domain text corpus and then adapted to our corpus with added pausing features. Results show that prosodic punctuation generation as a preliminary step to translation increases translation accuracy by 1\% in terms of BLEU scores. Encoding pauses as an extra feature during translation gives an additional 1\% increase to this number. We furthermore extend our translation system to jointly predict pause features in order to be used as an input to a text-to-speech system. We believe our results pave the way for expressive speech translation by inclusion of prosody in both encoding and decoding side of a speech translation system. 

The work carried out calls for recently popular neural network based systems that prove to be successful in real-world settings to consider a dimension necessary in human communication, such as prosody. This is done through both experimental and practical methodologies with results indicating that inclusion of prosody can indeed improve these systems.


\selectlanguage{catalan}
\vspace*{\fill}
\section*{\Large \sffamily  Resum}

En l’era actual en què augmenta considerablement la interacció dels humans amb les màquines, la prosòdia és un fenomen que rep poca atenció en sistemes que inclouen la comprensió del llenguatge natural. Els sistemes de reconeixement de veu se centren en el ``què s’ha dit?'' i no en el ``com s’ha dit?''. De la mateixa manera, els sistemes de traducció de la llengua oral que es construeixen sobre aquest tipus de reconeixement “ordinari”, no tenen en compte la informació prosòdia, la qual s’hauria de reflectir en la traducció i la síntesi de la llengua de destí. La prosòdia, que és la suma de característiques tals com l’entonació, l’accentuació i el ritme al llarg dels segments de veu, és un aspecte molt important en la comunicació humana oral. La prosòdia actua com un element essencial, des del marcatge de la importància en certs aspectes, fins a l’expressió d’emocions a l’ajuda en l’estructuració del discurs. Per tal d’arribar a un punt en què les màquines puguin escoltar-nos de la mateixa manera que ho fem els humans, la prosòdia com a element essencial del llenguatge parlat s’ha de tenir en compte.

En aquesta tesi estudiem la inclusió de la prosòdia en dues aplicacions que involucren la comprensió de la parla: la transcripció automàtica de la parla i la traducció de la llengua oral. Ens motiven especialment dos casos d’ús del món real: la subtitulació automàtica i el doblatge automàtic. Aquest casos d’ús prometen tenir un rol important en les futures aplicacions d’intel·ligència artificial, tenint en compte el creixement exponencial dels mitjans de comunicació. En el primer cas, proposem mètodes per millorar la precisió de la puntuació generada automàticament a través de l’ús de diverses característiques morfosintàctiques i prosòdiques. La puntuació, que s’ha demostrat que és essencial tant per a la comprensió humana com automàtica, normalmente no hi és a la sortida del reconeixement automàtic de la parla. Tenint en compte que tant la sintaxi com la prosòdia tenen una influència rellevant en la puntuació, primer mostrem l’efecte de cada característica prosòdica, a més a més de les característiques lèxiques, per a la generació de la puntuació en les transcripcions de veu. Els experiments mostren que el conjunt de característiques òptim és el que consisteix en les pauses entre paraules, l’entonació a nivell de paraula, i les categories gramatical (POS, part-of-speech). El nostre mètode, que utilitza un mecanisme d’atenció sobre seqüències paral·leles de característiques prosòdiques i morfosintàctiques, indica precisió de F1=70.3\% en la generació de la puntuació en transcripcions generades manualment. El sistema que proposem s’ha provat en un sistema de reconeixement de la parla d’última generació i mostra resultats prometedors. 

En el segon cas ens ocupem de la millora de la traducció de la llengua oral utilitzant la prosòdia. Presentem un nou mètode per a recopil·lar conjunts de dades bilingües anotades prosòdicament i l’utilitzem per a recopilar un corpus paral·lel anglès-castellà format per 7000 segments de veu alineats i transcrits de la versió original i la doblada d’una sèrie de televisió. A continuació construïm i entrenem un sistema neural de traducció automàtica amb un corpus de text més gran en el domini del cinema i després l’adaptem al nostre corpus amb característiques de pauses afegides. Els resultats mostren que la generació de puntuació prosòdica com a pas previ a la traducció augmenta la precisió de la traducció en un 1\% en termes de BLEU. La codificació de les pauses com a característica addicional encara incrementa la precisió en un altre 1\%. A més a més, també ampliem el nostre sistema de traducció per a predir conjuntament les característiques de pausa i poder-les utilitzar com a entrada en un sistema de síntesi de veu. Creiem que aquests resultats obren la porta a la traducció expressiva de la parla mitjançant la inclusió de la prosòdia tant en la codificació com en la descodificació del sistema de traducció oral.

El treball dut a terme fa evident la necessitat que els sistemes neuronals d’èxit més recents en el món real tinguin en compte una de les dimensions necessàries en la comunicació humana, com és la prosòdia. La manera de fer-ho és a través de metodologies experimentals i pràctiques i indica que la inclusió de la prosòdia pot millorar aquests sistemes.


\vspace*{\fill}



\selectlanguage{English}




%.\\

